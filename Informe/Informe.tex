\documentclass[11pt]{article}
\usepackage[a4paper, margin=2.54cm]{geometry}
\usepackage[utf8]{inputenc}
\usepackage[spanish, mexico]{babel}
\usepackage[spanish]{layout}
\usepackage[article]{ragged2e}
\usepackage{textcomp}
\usepackage{caption}
\usepackage{subcaption}
\usepackage{graphicx}
\usepackage{multirow}

% ============================================================================
% ============================================================================
% ============================================================================

\title{
  TRABAJO PRÁCTICO N° 1\\
  \large Estado de la arboleda en el sur de Buenos Aires
}
\author{
  Farizano, Juan Ignacio \\
  \and
  Mellino, Natalia
}
\date{}

% ============================================================================
% ============================================================================
% ============================================================================

\begin{document}

\maketitle
\newpage

\tableofcontents
\newpage

% ============================================================================
% ============================================================================
% ============================================================================

\section{Introducción}
\textbf{Motivación del problema:}

\begin{justify}
  En el año 2011 se realizó un Censo Forestal Urbano Público en dos
  comunas del sur de Buenos Aires y esta será nuestra fuente de información
  a lo largo del informe.
  En éste analizaremos los datos de dicho censo con el objetivo
  de determinar el estado actual del arbolado urbano público.\\
  Las variables incluidas en la base de datos dada se describen a continuación.
\end{justify}

% Tabla de variables
\begin{center}
  \begin{tabular}{| c | p{9cm} | c |}
    \hline
    \textbf{Nombre} & \textbf{Descripción} & \textbf{Tipo de variable} \\ \hline
    ID & Identificación del árbol. & - \\ \hline
    Altura & Altura de cada árbol, medida en metros (m). Observación: si un árbol
    mide 12,7 m se tomará como dato “12”, truncando los valores a la unidad. & 
    Cuantitativa continua \\ \hline
    Diámetro & Diámetro de cada árbol, medido en centímetros (cm). & 
    Cuantitativa continua \\ \hline
    Inclinación & Ángulo que forma el tronco del árbol respecto
    a una perpendicular al suelo, medido en grados (°).
    Indica el grado de inclinación del árbol. & Cuantitativa continua \\ \hline
    Especie & Especie a la que pertenece el árbol, dentro de las siguientes
    categorías: Eucalipto, Jacarandá, Palo Borracho, Casuarina, Fresno, Ceibo,
    Ficus, Álamo, Acacia. & Cualitativa nominal \\ \hline
    Origen & Procedencia de la especie: Exótico, Nativo/Autóctono, 
    No Determinado. & Cualitativa nominal \\ \hline
    Brotes & Número de brotes jóvenes crecidos durante el último año. &
    Cuantitativa discreta \\ \hline
  \end{tabular}
\end{center}

% ============================================================================
% ============================================================================
% ============================================================================

\newpage
\section{Análisis univariado}

\subsection{Altura}

\begin{table}[h!]
  \begin{center}
    \caption*{\textbf{Altura de los árboles en metros}}
    \begin{tabular}{| c | r | r | r | r |}
      \hline
      \multirow{3}{3cm}{\centering\textbf{Intervalos \\ (en m)}} & 
      \multirow{3}{2.5cm}{\centering\textbf{Frecuencia absoluta}} & 
      \multirow{3}{2.5cm}{\centering\textbf{Frecuencia absoluta acumulada}} &
      \multirow{3}{2.5cm}{\centering\textbf{Frecuencia relativa}} & 
      \multirow{3}{2.5cm}{\centering\textbf{Frecuencia relativa acumulada}} \\
      & & & & \\
      & & & & \\ \hline
      (0,5] & 34 & 34 & 0.0971 & 0.0971 \\ \hline
      (5,10] & 79 & 113 & 0.2257 & 0.3229 \\ \hline
      (10,15] & 92 & 205 & 0.2629 & 0.5857 \\ \hline
      (15,20] & 84 & 289 & 0.2400 & 0.8257 \\ \hline
      (20,25] & 53 & 342 & 0.1514 & 0.9771 \\ \hline
      (25,30] & 5 & 347 & 0.0143 & 0.9914 \\ \hline
      (30,35] & 3 & 350 & 0.0086 & 1.0000 \\ \hline
      \textbf{Total} & \textbf{350} & & \textbf{1} & \\ \hline
    \end{tabular}
    \caption{}
    \label{tab:tablaAltura}
  \end{center}
\end{table}

\begin{figure}[h!]
  \begin{center}
    \begin{subfigure}[b]{0.9\linewidth}
      \includegraphics[width=\linewidth]{histAltura.pdf}
      \caption{}
      \label{fig:histAltura}
    \end{subfigure}
  \end{center}
\end{figure}

\newpage

\begin{figure}[h!]
  \ContinuedFloat
  \begin{center}
    \begin{subfigure}[b]{0.9\linewidth}
      \includegraphics[width=\linewidth]{acumAltura.pdf}
      \caption{}
      \label{fig:acumAltura}
    \end{subfigure}
    \caption{}
  \end{center}
\end{figure}

\begin{table}[h!]
  \begin{center}
    \begin{tabular}{| c | r |}
      \hline
      \textbf{Medida (en m)} & \textbf{Valor} \\ \hline
      \textbf{Mínimo} & 1.00 \\ \hline
      \textbf{Primer Cuartil} & 9.00 \\ \hline
      \textbf{Mediana} & 14.00 \\ \hline
      \textbf{Media} & 14.02 \\ \hline
      \textbf{Tercer Cuartil} & 18.00 \\ \hline
      \textbf{Máximo} & 35.00 \\ \hline
    \end{tabular}
    \caption{Medidas descriptivas.}
    \label{tab:descripAltura}
  \end{center}
\end{table}

\begin{justify}
  El gráfico \ref{fig:histAltura} es unimodal y asimétrico hacia la derecha,
  donde se puede observar que el intervalo modal es de 10 a 15 mts.
  Si se observa la figura \ref{fig:acumAltura}
  junto con la tabla \ref{tab:tablaAltura} se puede ver que aproximadamente
  el 97\% de la arboleda presenta una altura inferior a 25 mts.
  La altura promedio es de 14.02mts.
\end{justify}

% ============================================================================

\newpage
\subsection{Diámetro}

\begin{table}[h!]
  \begin{center}
    \caption*{\textbf{Diámetro de los árboles en centímetros}}
    \begin{tabular}{| c | r | r | r | r |}
      \hline
      \multirow{3}{3cm}{\centering\textbf{Intervalos \\ (en cm)}} & 
      \multirow{3}{2.5cm}{\centering\textbf{Frecuencia absoluta}} & 
      \multirow{3}{2.5cm}{\centering\textbf{Frecuencia absoluta acumulada}} &
      \multirow{3}{2.5cm}{\centering\textbf{Frecuencia relativa}} & 
      \multirow{3}{2.5cm}{\centering\textbf{Frecuencia relativa acumulada}} \\
      & & & & \\
      & & & & \\ \hline
      (0,20] & 83 & 83 & 0.2371 & 0.2371 \\ \hline
      (20,40] & 147 & 230 & 0.4200 & 0.6571 \\ \hline
      (40,60] & 69 & 299 & 0.1971 & 0.8543 \\ \hline
      (60,80] & 35 & 334 & 0.1000 & 0.9543 \\ \hline
      (80,100] & 12 & 346 & 0.0343 & 0.9886 \\ \hline
      (100,120] & 2 & 348 & 0.0057 & 0.9943 \\ \hline
      (120,140] & 2 & 350 & 0.0057 & 1.0000 \\ \hline
      \textbf{Total} & \textbf{350} & & \textbf{1} & \\ \hline
    \end{tabular}
    \caption{}
    \label{tab:tablaDiametro}
  \end{center}
\end{table}

\begin{figure}[h!]
  \begin{center}
    \begin{subfigure}[b]{0.9\linewidth}
      \includegraphics[width=\linewidth]{histDiametro.pdf}
      \caption{}
      \label{fig:histDiametro}
    \end{subfigure}
  \end{center}
\end{figure}

\begin{figure}[h!]
  \ContinuedFloat
  \begin{center}
    \begin{subfigure}[b]{0.9\linewidth}
      \includegraphics[width=\linewidth]{acumDiametro.pdf}
      \caption{}
      \label{fig:acumDiametro}
    \end{subfigure}
    \caption{}
  \end{center}
\end{figure}

\newpage

\begin{table}[h!]
  \begin{center}
    \begin{tabular}{| c | r |}
      \hline
      \textbf{Medida (en cm)} & \textbf{Valor} \\ \hline
      \textbf{Mínimo} & 1.00 \\ \hline
      \textbf{Primer Cuartil} & 22.00 \\ \hline
      \textbf{Mediana} & 32.00 \\ \hline
      \textbf{Media} & 37.61 \\ \hline
      \textbf{Tercer Cuartil} & 50.00 \\ \hline
      \textbf{Máximo} & 135.00 \\ \hline
    \end{tabular}
    \caption{Medidas descriptivas.}
    \label{tab:descripDiametro}
  \end{center}
\end{table}

\begin{justify}
  Como se puede observar en el gráfico de la figura \ref{fig:histDiametro},
  éste es asimétrico a la derecha, mostrando que el intervalo modal
  es de 20 a 40 cm. El diámetro promedio resultó ser
  de 37.61 cm. Observando la figura
  \ref{fig:acumDiametro} se puede comprobar que aproximadamente el 80\%
  de los árboles tienen un diámetro menor a 60cm.
\end{justify}

% ============================================================================

\newpage
\subsection{Inclinación}

\begin{table}[h!]
  \begin{center}
    \caption*{\textbf{Inclinación de los árboles en grados}}
    \begin{tabular}{| c | r | r | r | r |}
      \hline
      \multirow{3}{3cm}{\centering\textbf{Intervalos (inclinación en grados)}} & 
      \multirow{3}{2.5cm}{\centering\textbf{Frecuencia absoluta}} & 
      \multirow{3}{2.5cm}{\centering\textbf{Frecuencia absoluta acumulada}} &
      \multirow{3}{2.5cm}{\centering\textbf{Frecuencia relativa}} & 
      \multirow{3}{2.5cm}{\centering\textbf{Frecuencia relativa acumulada}} \\
      & & & & \\
      & & & & \\ \hline
      [0,7) & 294 & 294 & 0.8400 & 0.8400 \\ \hline
      [7,14) & 34 & 328 & 0.0971 & 0.9371 \\ \hline
      [14,21) & 13 & 341 & 0.0371 & 0.9743 \\ \hline
      [21,28) & 4 & 345 & 0.0114 & 0.9857 \\ \hline
      [28,35) & 2 & 347 & 0.0057 & 0.9914 \\ \hline
      [35,42) & 3 & 350 & 0.0086 & 1.0000 \\ \hline
      \textbf{Total} & \textbf{350} & & \textbf{1} & \\ \hline
    \end{tabular}
    \caption{}
    \label{tab:tablaInclinacion}
  \end{center}
\end{table}

\begin{figure}[h!]
  \begin{center}
    \includegraphics[width=0.9\linewidth]{boxInclinacion.pdf}
    \caption{}
    \label{fig:boxInclinacion}
  \end{center}
\end{figure}

\newpage

\begin{table}[h!]
  \begin{center}
    \begin{tabular}{| c | r |}
      \hline
      \textbf{Medida (en °)} & \textbf{Valor} \\ \hline
      \textbf{Mínimo} & 0.00 \\ \hline
      \textbf{Primer Cuartil} & 0.00 \\ \hline
      \textbf{Mediana} & 0.00 \\ \hline
      \textbf{Media} & 2.78 \\ \hline
      \textbf{Tercer Cuartil} & 3.00 \\ \hline
      \textbf{Máximo} & 40.00 \\ \hline
    \end{tabular}
    \caption{Medidas descriptivas.}
    \label{tab:descripInclinacion}
  \end{center}
\end{table}

\begin{justify}
  En el gráfico \ref{fig:boxInclinacion} se observa que la mediana
  es de 0 grados y además observando la tabla \ref{tab:tablaInclinacion}
  se ve que los valores adyacentes que van de una inclinación de 0 a 7 grados,
  siendo este el intervalo modal, componen el 84\% de la arboleda.
  Fuera de este rango, se encuentran 52 observaciones outliers.
\end{justify}

% ============================================================================

\subsection{Especie}

\begin{table}[h!]
  \begin{center}
    \caption*{\textbf{Cantidad de árboles por especie}}
    \begin{tabular}{| c | r |}
      \hline
      \textbf{Especie} & \textbf{Frecuencia absoluta} \\ \hline
      Acacia & 25 \\ \hline
      Álamo & 59 \\ \hline
      Casuarina & 55 \\ \hline
      Ceibo & 17 \\ \hline
      Eucalipto & 67 \\ \hline
      Ficus & 12 \\ \hline
      Fresno & 30 \\ \hline
      Jacarandá & 44 \\ \hline
      Palo borracho & 41 \\ \hline
      \textbf{Total} & \textbf{350} \\ \hline
    \end{tabular}
    \caption{}
    \label{tab:tablaEspecie}
  \end{center}
\end{table}

\newpage

\begin{figure}[h!]
  \begin{center}
    \includegraphics[width=0.9\linewidth]{barrasEspecie.pdf}
    \caption{}
    \label{fig:barrasEspecie}
  \end{center}
\end{figure}

\begin{justify}
  Si se observa el gráfico de la figura \ref{fig:barrasEspecie},
  se puede ver que la especie con más ejemplares en el censo resultó
  ser el eucalipto con una cantidad total de 67, seguido por Álamo
  con 59 y Casuarina con 55.
\end{justify}

% ============================================================================

\newpage
\subsection{Origen}

\begin{table}[h!]
  \begin{center}
    \caption*{\textbf{Origen de los árboles}}
    \begin{tabular}{| c | r | r |}
      \hline
      \textbf{Especie} & \textbf{Frecuencia absoluta} & 
      \textbf{Frecuencia relativa} \\ \hline
      Exótico	& 241	& 0.6886 \\ \hline
      Nativo/Autóctono & 109 & 0.3114 \\ \hline
      \textbf{Total} & \textbf{350} & \textbf{1} \\ \hline
    \end{tabular}
    \caption{}
    \label{tab:tablaOrigen}
  \end{center}
\end{table}

\begin{figure}[h!]
  \begin{center}
    \includegraphics[width=0.9\linewidth]{pieOrigen.pdf}
    \caption{}
    \label{fig:pieOrigen}
  \end{center}
\end{figure}

\begin{justify}
  Observando la figura \ref{fig:pieOrigen} se puede ver que el 68.86\% de los
  árboles es de origen exótico y 31.14\% de origen Nativo/Autóctono.
\end{justify}

% ============================================================================

\newpage
\subsection{Brotes}

\begin{table}[h!]
  \begin{center}
    \caption*{\textbf{Árboles por su cantidad de brotes}}
    \begin{tabular}{| c | r | r | r | r |}
      \hline
      \multirow{3}{3cm}{\centering\textbf{Cantidad de brotes}} & 
      \multirow{3}{2.5cm}{\centering\textbf{Frecuencia absoluta}} & 
      \multirow{3}{2.5cm}{\centering\textbf{Frecuencia absoluta acumulada}} &
      \multirow{3}{2.5cm}{\centering\textbf{Frecuencia relativa}} & 
      \multirow{3}{2.5cm}{\centering\textbf{Frecuencia relativa acumulada}} \\
      & & & & \\
      & & & & \\ \hline
      0	& 1	& 1 & 0.0029	& 0.0029 \\ \hline
      1	& 16	& 17	& 0.0457	& 0.0486 \\ \hline
      2	& 65	& 82	& 0.1857	& 0.2343 \\ \hline
      3	& 100	& 182	& 0.2857	& 0.5200 \\ \hline
      4	& 87	& 269	& 0.2486	& 0.7686 \\ \hline
      5	& 47	& 316	& 0.1343	& 0.9029 \\ \hline
      6	& 25	& 341	& 0.0714	& 0.9743 \\ \hline
      7	& 7	& 348	& 0.0200	& 0.9943 \\ \hline
      8	& 2	& 350	& 0.0057	& 1.0000 \\ \hline
      \textbf{Total} & \textbf{350} & & \textbf{1} & \\ \hline
    \end{tabular}
    \caption{}
    \label{tab:tablaBrotes}
  \end{center}
\end{table}

\begin{figure}[h!]
  \begin{center}
    \begin{subfigure}[b]{0.9\linewidth}
      \includegraphics[width=\linewidth]{bastonesBrotes.pdf}
      \caption{}
      \label{fig:bastonesBrotes}
    \end{subfigure}
  \end{center}
\end{figure}

\newpage

\begin{figure}[h!]
  \ContinuedFloat
  \begin{center}
    \begin{subfigure}[b]{0.9\linewidth}
      \includegraphics[width=\linewidth]{acumBrotes.pdf}
      \caption{}
      \label{fig:acumBrotes}
    \end{subfigure}
    \caption{}
  \end{center}
\end{figure}

\begin{table}[h!]
  \begin{center}
    \begin{tabular}{| c | r |}
      \hline
      \textbf{Medida (en cm)} & \textbf{Valor} \\ \hline
      \textbf{Mínimo} & 1.00 \\ \hline
      \textbf{Primer Cuartil} & 22.00 \\ \hline
      \textbf{Mediana} & 32.00 \\ \hline
      \textbf{Media} & 37.61 \\ \hline
      \textbf{Tercer Cuartil} & 50.00 \\ \hline
      \textbf{Máximo} & 135.00 \\ \hline
    \end{tabular}
    \caption{Medidas descriptivas.}
    \label{tab:descripDiametro}
  \end{center}
\end{table}

\begin{justify}
  En el gráfico de la figura \ref{fig:bastonesBrotes} se observa que
  existe una mayor cantidad de árboles que presentaron 3 brotes crecidos
  durante el último año. La cantidad de brotes promedio resultó ser 3.5
  con una desviación estandar de 1.40.
\end{justify}

% ============================================================================
% ============================================================================
% ============================================================================

\newpage
\section{Análisis bivariado}

\subsection{Altura/Especie}

\begin{table}[h!]
  \begin{center}
    \caption*{\textbf{Altura de los árboles (en metros) según su especie}}
    \begin{tabular}{| c | r | r | r | r | r | r |}
      \hline
      \textbf{Especie} & \textbf{Mínimo} & \textbf{Máximo} & \textbf{Promedio} &
      \textbf{Mediana} & \textbf{Primer cuartil} & \textbf{Tercer cuartil}  \\ \hline
      Acacia & 3 & 19 & 10.7200 & 12 & 6.00 & 14.00 \\ \hline
	    Álamo & 4 & 35 & 15.3729 & 15 & 12.00 & 18.00 \\ \hline
	    Casuarina & 5 & 24 & 14.8909 & 15 & 11.50 & 18.50 \\ \hline
	    Ceibo & 2 & 17 & 7.6471 & 6 & 5.00 & 10.00 \\ \hline
	    Eucalipto & 1 & 35 & 19.4627 & 21 & 17.00 & 22.00 \\ \hline
	    Ficus & 1 & 18 & 9.0833 & 10 & 5.75 & 11.50 \\ \hline
	    Fresno & 5 & 22 & 10.7333 & 10 & 8.00 & 12.00 \\ \hline
	    Jacarandá & 3 & 25 & 12.2045 & 13 & 7.00 & 16.25 \\ \hline
	    Palo borracho & 3 & 24 & 12.4390 & 12 & 9.00 & 16.00 \\ \hline
    \end{tabular}
    \caption{}
    \label{tab:tablaAlturaEspecie}
  \end{center}
\end{table}

\begin{figure}[h!]
  \begin{center}
    \includegraphics[width=0.9\linewidth]{boxAlturaEspecie.pdf}
    \caption{}
    \label{fig:boxAlturaEspecie}
  \end{center}
\end{figure}

\begin{justify}
  En el gráfico de la figura \ref{fig:boxAlturaEspecie} se observa
  que la especie Eucalipto es la que presenta una mayor altura con observaciones
  que alcanzan los 35m de altura, sin embargo, se han observado valores extremos
  de esta especie que presentan una altura muy baja, siendo la más baja de 1 metro.
  Con respecto a las demás, se puede ver que no presentan observaciones extremas
  a excepción del Álamo y el Fresno donde se observan unas pocas.
  Por otro lado, si se observa la tabla \ref{tab:tablaAlturaEspecie} junto con el
  gráfico, se ve que a excepción del Eucalipto y el Ceibo, las medianas de las
  demás especies no difieren mucho unas de otras.
\end{justify}

% ============================================================================

\subsection{Diámetro/Especie}

\begin{table}[h!]
  \begin{center}
    \caption*{\textbf{Diámetro de los árboles (en centímetros) según su especie}}
    \begin{tabular}{| c | r | r | r | r | r | r |}
      \hline
      \textbf{Especie} & \textbf{Mínimo} & \textbf{Máximo} & \textbf{Promedio} &
      \textbf{Mediana} & \textbf{Primer cuartil} & \textbf{Tercer cuartil}  \\ \hline
      Acacia & 4 & 83 & 30.9600 & 24 & 18.00 & 38.00 \\ \hline
      Álamo & 3 & 9 & 	26.7966 & 26 & 14.50 & 34.00 \\ \hline
      Casuarina & 1 & 83 & 42.7091 & 45 & 24.50 & 60.00 \\ \hline
      Ceibo & 5 & 68 & 28.2941 & 26 & 20.00 & 30.00 \\ \hline
      Eucalipto & 20 & 135 & 49.1642 & 38 & 32.00 & 57.50 \\ \hline
      Ficus & 7 & 46 & 21.8333 & 16 & 13.50 & 27.50 \\ \hline
      Fresno & 12 & 72 & 29.6667 & 24 & 19.25 & 32.75 \\ \hline
      Jacarandá & 3 & 97 & 34.3182 & 33 & 20.00 & 46.25 \\ \hline
      Palo borracho & 6 & 130 & 49.3415 & 50 & 36.00 & 64.00 \\ \hline
    \end{tabular}
    \caption{}
    \label{tab:tablaDiametroEspecie}
  \end{center}
\end{table}

\begin{figure}[h!]
  \begin{center}
    \includegraphics[width=0.9\linewidth]{boxDiametroEspecie.pdf}
    \caption{}
    \label{fig:boxDiametroEspecie}
  \end{center}
\end{figure}

\begin{justify}
  Observando la tabla \ref{tab:tablaDiametroEspecie} y el gráfico
  de la figura \ref{fig:boxDiametroEspecie}, se observa que el Eucalipto es la
  especie que presenta mayor diámetro donde sus valores adyacentes se
  aproximan a los 100cm, aunque se observan algunos valores extremos que llegan
  a alcanzar los 135cm de diámetro. Por otro lado, el Ficus resultó ser la
  especie con menor diámetro, donde sus ejemplares apenas alcanzan los 46cm.
  Con respecto a las medianas, se observa en la tabla que las especies
  Acacia, Álamo, Ceibo y Fresno presentan medianas similares mientras que el resto
  difieren notablemente.
  En general, se puede ver en el gráfico que todas las especies presentan valores
  extremos.
\end{justify}

% ============================================================================

\subsection{Inclinación/Especie}

\begin{table}[h!]
  \begin{center}
    \caption*{\textbf{Inclinación de los árboles (en grados) según su especie}}
    \begin{tabular}{| c | r | r | r | r | r | r |}
      \hline
      \textbf{Especie} & \textbf{Mínimo} & \textbf{Máximo} & \textbf{Promedio} &
      \textbf{Mediana} & \textbf{Primer cuartil} & \textbf{Tercer cuartil}  \\ \hline
      Acacia & 0 & 15 & 3.4000 & 0 & 0 & 6.0 \\ \hline
      Álamo & 0 & 3 & 0.1017 & 0 & 0 & 0.0 \\ \hline
      Casuarina & 0 & 25 & 1.1273 & 0 & 0 & 0.0 \\ \hline
      Ceibo & 0 & 35 & 7.7059 & 0 & 0 & 11.0 \\ \hline
      Eucalipto & 0 & 40 & 3.4328 & 0 & 0 & 1.5 \\ \hline
      Ficus & 0 & 4 & 0.3333 & 0 & 0 & 0.0 \\ \hline
      Fresno & 0 & 15 & 2.7000 & 0 & 0 & 4.0 \\ \hline
      Jacarandá & 0 & 25 & 6.6136 & 6 & 0 & 10.0 \\ \hline
      Palo borracho & 0 & 15 & 2.0488 & 0 & 0 & 3.0 \\ \hline
    \end{tabular}
    \caption{}
    \label{tab:tablaInclinacionEspecie}
  \end{center}
\end{table}

\begin{figure}[h!]
  \begin{center}
    \includegraphics[width=0.9\linewidth]{boxInclinacionEspecie.pdf}
    \caption{}
    \label{fig:boxInclinacionEspecie}
  \end{center}
\end{figure}

\begin{justify}
  Observando la tabla \ref{tab:tablaInclinacionEspecie} junto con el
  gráfico de la figura \ref{fig:boxInclinacionEspecie}, se puede ver que el
  Jacarandá fue la especie cuyos valores adyacentes presentaron una mayor inclinación
   alcanzando una inclinación máxima de 25°. Por el contrario, el Álamo 
   esultó ser la especie con menor inclinación, donde todos sus ejemplares no superan los 0°,
  a excepción de una observación extrema que alcanzó una inclinación de 3°.
  En general, todas las especies, a excepción del Jacarandá presentan una mediana
  de 0°. Por otro lado, la especies Eucalipto, Casuarina y Ceibo fueron las que
  presentaron más valores extremos.
\end{justify}

% ============================================================================

\subsection{Origen/Especie}

\begin{table}[h!]
  \begin{center}
    \caption*{\textbf{Origen de los árboles según su especie}}
    \begin{tabular}{| c | r | r | r |}
      \hline
      \textbf{Especie} & \textbf{Exótico} & \textbf{Nativo/Autóctono} &
      \textbf{Total} \\ \hline
      Acacia & 18 & 7 & \textbf{25} \\ \hline
      Álamo & 59 & 0 & \textbf{59} \\ \hline
      Casuarina & 55 & 0 & \textbf{55} \\ \hline
      Ceibo & 0 & 17 & \textbf{17} \\ \hline
      Eucalipto & 67 & 0 & \textbf{67} \\ \hline
      Ficus & 12 & 0 & \textbf{12} \\ \hline
      Fresno & 30 & 0 & \textbf{30} \\ \hline
      Jacarandá & 0 & 44 & \textbf{44} \\ \hline
      Palo borracho & 0 & 41 & \textbf{41} \\ \hline
      \textbf{Total} & \textbf{241} & \textbf{109} & \textbf{350} \\ \hline
    \end{tabular}
    \caption{}
    \label{tab:tablaOrigenEspecie}
  \end{center}
\end{table}

\begin{figure}[h!]
  \begin{center}
    \includegraphics[width=0.9\linewidth]{barrasOrigenEspecie.pdf}
    \caption{}
    \label{fig:barrasOrigenEspecie}
  \end{center}
\end{figure}

\newpage

\begin{justify}
  En la figura \ref{fig:barrasOrigenEspecie} se puede notar fácilmente
  las especies que son exclusivamente nativas o exóticas, excepto por la Acacia
  que es tanto nativa como exótica. La mayoria de las especies resultan ser
  de origen exótico.
\end{justify}

% ============================================================================
% ============================================================================
% ============================================================================

\section{Conclusiones}

\begin{itemize}
  \item La especie que más abunda es el Eucalipto, seguido del Álamo y la Casuarina
        (ver sección 2.4).
  \item Dentro de la muestra se encuentra mayor cantidad de ejemplares de origen
        exótico que nativo/autóctono (ver sección 2.5). Al analizar el origen
        según la especie, como se hizo en la sección 3.4, también se concluyó
        que la mayoría de las especies son de origen exótico exclusivamente.
  \item Con respecto a la cantidad de brotes, a excepción de un sólo ejemplar, el
        resto sí presenta brotes crecidos durante el último año (ver sección 2.6.).
  \item El eucalipto es la especie que presenta mayor altura y diámetro con
        respecto a las demás (ver secciones 3.1 y 3.2 respectivamente).
  \item La mayoría de los árboles de la muestra no presentan inclinación. Esto
        lo podemos leer en detalle, en la sección 3.3.
\end{itemize}

\end{document}
